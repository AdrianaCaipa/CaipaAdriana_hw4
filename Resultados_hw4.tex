\documentclass{article}
\usepackage[utf8]{inputenc}
\usepackage{float}
\usepackage{graphicx}

\title{Metodos Computacionales. Tarea 4}
\author{Adriana Lucia Caipa Furque}
\date{November 2018}

\begin{document}

\maketitle
\section{ODE}
\subsection{Angulo de 45}
\begin{figure}[H] 
\centering
\includegraphics[width=0.6\textwidth]{Adriana_CaipaGrafica45.png}
\caption{\label{fig:data}Proyectil a 45 grados}
\end{figure}
En esta gráfica se puede observar como el proyectil a un angulo de 45 grados, a medida que aumenta la trayectoria va aumentando la altura. Hasta que se llega a la altura maxima y empieza a decaer de forma muy rapida. Asimismo, se puede ver como la trayectoria que recorre no supera los 5 metros, esto se debe al efecto de la friccion sobre el cuerpo. 
\section{PDE}
\subsection{Fronteras fijas}

\begin{figure}[H] 
\centering
\includegraphics[width=0.6\textwidth]{Promedio_f.png}
\caption{\label{fig:data}Promedio fronteras fijas}
\end{figure}
Se puede ver en esta grafica como alcanza la temperatura maxima de 24 rados. Se observa como esta va aumentando a medida que el tiempo pasa. Hasta que alcanza el equilibrio.
\begin{figure}[H] 
\centering
\includegraphics[width=0.6\textwidth]{Temperatura_estadof.png}
\caption{\label{fig:data}Temperatura estado intermedio 1}
\end{figure}

\begin{figure}[H] 
\centering
\includegraphics[width=0.6\textwidth]{Temperatura_estado2f.png}
\caption{\label{fig:data}Temperatura estado intermedio 2 }
\end{figure}
En la figura 3 y 4 se ve como se mantienen similares ambas graficas, ya que, las fronteras se mantienen siempre iguales, haciendo que se alcance el equilibrio.

\begin{figure}[H] 
\centering
\includegraphics[width=0.6\textwidth]{Temperatura_estadofinf.png}
\caption{\label{fig:data}Temperatura estado intermedio 2 }
\end{figure}


\subsection{Fronteras Periodicas}

\begin{figure}[H] 
\centering
\includegraphics[width=0.6\textwidth]{Promedio_p.png}
\caption{\label{fig:data}Promedio fronteras periodicas}
\end{figure}

\begin{figure}[H] 
\centering
\includegraphics[width=0.6\textwidth]{Temperatura_estado1p.png}
\caption{\label{fig:data}Temperatura estado intermedio 1 }
\end{figure}

\begin{figure}[H] 
\centering
\includegraphics[width=0.6\textwidth]{Temperatura_estado2p.png}
\caption{\label{fig:data}Temperatura estado intermedio 2}
\end{figure}

\begin{figure}[H] 
\centering
\includegraphics[width=0.6\textwidth]{Temperatura_estadofinp.png}
\caption{\label{fig:data}Temperatura estado intermedio 2 }
\end{figure}
\subsection{Fronteras abiertas}

\begin{figure}[H] 
\centering
\includegraphics[width=0.6\textwidth]{Promedio_l.png}
\caption{\label{fig:data}Promedio fronteras libres}
\end{figure}

\begin{figure}[H] 
\centering
\includegraphics[width=0.6\textwidth]{Temperatura_estadol.png}
\caption{\label{fig:data}Temperatura estado intermedio libre 1}
\end{figure}

\begin{figure}[H] 
\centering
\includegraphics[width=0.6\textwidth]{Temperatura_estado2l.png}
\caption{\label{fig:data}Temperatura estado intermedio libre 2}
\end{figure}

\begin{figure}[H] 
\centering
\includegraphics[width=0.6\textwidth]{Temperatura_finl.png}
\caption{\label{fig:data}Temperatura estado intermedio 2 }
\end{figure}

\end{document}

